\documentclass{ctexart}
\title{Experiment Report Template}
\author{<Name>}
\usepackage{xeCJK}
% \setCJKmainfont[BoldFont=STSongti-SC-Bold]{STSong}
\renewcommand\contentsname{Contents}
\usepackage[left = 2cm , right = 2cm , top = 3cm, bottom = 3cm]{geometry}
\usepackage{amsmath}
\numberwithin{equation}{section}% 公式按章节分
\usepackage{amssymb}
\usepackage{footnote}
\usepackage{yhmath}
\usepackage{bm}
\usepackage{float}
\usepackage{setspace}
\linespread{1.5}
\usepackage{subfigure}
\usepackage{booktabs} %\toprule
\usepackage{array}% tabular using m{2cm}, m , vertical alignment
\usepackage{titlesec}
\usepackage{appendix}
\usepackage{graphicx}
\usepackage{wrapfig}
\usepackage{caption2}
\usepackage{tikz}
\usetikzlibrary{arrows.meta}
\usepackage[T1]{fontenc}
\usepackage{tabularx}
%定义插入 TikZ 作图命令
\newcommand{\inputtikzfigure}[1]{\input{#1.tex}}
%定义罗马数学
\usepackage{mathrsfs}%使用花体字 命令\mathscr
\usepackage[bookmarks=true,colorlinks,linkcolor=blue,anchorcolor=blue,citecolor=green]{hyperref} %超链接

% 页眉, 表头信息
\newcommand{\ExpTitle}{实验<XXN> <ExperimentName>}
\newcommand{\stuName}{<Name>}
\newcommand{\stuID}{<Student ID>}
\newpagestyle{SPAhead}{            
    \sethead{中山大学物理与天文学院基础物理实验预习报告}{$\quad$\stuName}{\ExpTitle}     %设置页眉
    \setfoot{}{}{}      %设置页脚,可以在页脚添加 
    % \thepage           % 显示页数
    \headrule           % 添加页眉的下划线
    % \footrule          % 添加页脚的下划线
}
\pagestyle{SPAhead}    %使用该style

% ================ 回答环境 ===================
\usepackage{amsthm}
\usepackage[framemethod=tikz]{mdframed}

\definecolor{ocre}{RGB}{243,102,25}
\definecolor{mygray}{RGB}{243,243,244}
\definecolor{myblue}{RGB}{18,24,176}

\newcommand*\mymathbox[1]{%
    \fcolorbox{ocre}{mygray}{\hspace{1em}#1\hspace{1em}}}

\newtheoremstyle{ansstyle}
    {\topsep}
    {\topsep}
    {\normalfont}
    {}
    {\sffamily\bfseries}
    {\\}
    {.5em}
    {{\color{ocre}\thmname{#1}~\thmnumber{#2}}\thmnote{\,--\,#3}}%

\theoremstyle{ansstyle}
% \theoremstyle{break}  
\newmdtheoremenv[
    backgroundcolor=mygray,
    linecolor=ocre,
    leftmargin=0pt,
    innerleftmargin=10pt,
    innerrightmargin=10pt,
    ]{ans}{Answer}[section]

%%%%%%%%%%%%%%%%%%%%%%%%%%%%%%%%%%%%%%%%%%%%%%%%%%%%
% -------------------- DOCUMENT --------------------
%%%%%%%%%%%%%%%%%%%%%%%%%%%%%%%%%%%%%%%%%%%%%%%%%%%%

\begin{document}
% \maketitle

\noindent
\renewcommand\arraystretch{1.8}
\begin{center}
    \begin{tabular}{|p{1.49cm}<{\centering}|p{1.49cm}<{\centering}|p{1.49cm}<{\centering}|p{1.49cm}<{\centering}|p{1.49cm}<{\centering}|p{1.49cm}<{\centering}|p{1.49cm}<{\centering}|p{1.49cm}<{\centering}|}
        \specialrule{0em}{0.3cm}{0cm}
        \hline
        \multicolumn{2}{|c|}{\LARGE 预习实验} & \multicolumn{2}{c|}{\LARGE 实验记录}& \multicolumn{2}{c|}{\LARGE 分析讨论} & \multicolumn{2}{c|}{\LARGE 总成绩} \\
        \hline
        &&&&&&& \\
        \hline
        \specialrule{0em}{0.3cm}{0cm}
    \end{tabular}
\end{center}

% \noindent
\renewcommand\arraystretch{1.3}
\begin{center}
    \begin{tabular}{|p{1.5cm}|p{4.5cm}|p{4cm}|p{3.65cm}|}
        \hline
        {\large 专业}: & {\large 物理学}    & {\large 年级:}    & {\large <Grade>} \\
        \hline
        {\large 姓名:} & {\large \stuName} & {\large 学号:}    & {\large \stuID} \\
        \hline
        {\large 日期:} & {\large         } & {\large 教师签名:} & \\
        \hline
        \specialrule{0em}{0.6cm}{0cm}
    \end{tabular}
\end{center}

\begin{center}
    \LARGE{\textbf{\ExpTitle}}
\end{center}
    
\large{\textbf{【实验报告注意事项】}} %-----【实验报告注意事项】-----
    
\begin{enumerate}
    \item 实验报告由三部分组成:
    \begin{enumerate}
        \item \textbf{预习报告}:(提前一周)认真研读\underline{\textbf{实验讲义}},弄清实验原理;实验所需的仪器设备、用具及其使用(强烈建议到实验室预习),完成讲义中的预习思考题;了解实验需要测量的物理量,并根据要求提前准备实验记录表格(由学生自己在实验前设计好,可以打印)。预习成绩低于10分(共20分)者不能做实验。
        \item \textbf{实验记录}:认真、客观记录实验条件、实验过程中的现象以及数据。实验记录请用珠笔或者钢笔书写并签名({\color{red}用铅笔记录的被认为无效})。{\color{red}保持原始记录,包括写错删除部分,如因误记需要修改记录,必须按规范修改。}(不得输入电脑打印,但可扫描手记后打印扫描件);离开前请实验教师检查记录并签名。
        \item \textbf{分析讨论}:处理实验原始数据(学习仪器使用类型的实验除外),对数据的可靠性和合理性进行分析;按规范呈现数据和结果(图、表),包括数据、图表按顺序编号及其引用;分析物理现象(含回答实验思考题,写出问题思考过程,必要时按规范引用数据);最后得出结论。
    \end{enumerate}
    \underline{实验报告}就是预习报告、实验记录、和数据处理与分析合起来,加上本页封面。
    \item 每次完成实验后的一周内交\underline{实验报告}。
    \item 除实验记录外,实验报告其他部分建议双面打印。
\end{enumerate}




% !!!!!============================!!!!!!!============================!!!!!
% !!!!!~~~~~~~~~~~~~~~~~~~~~~~~~~~~~~~~~~~~~~~~~~~~~~~~~~~~~~~~~~~~~~~!!!!!
% !!!!!--------------------------- 预习报告 ---------------------------!!!!!
% !!!!!~~~~~~~~~~~~~~~~~~~~~~~~~~~~~~~~~~~~~~~~~~~~~~~~~~~~~~~~~~~~~~~!!!!!
% !!!!!============================!!!!!!!============================!!!!!
\newpage
\tableofcontents

\begin{center}
    \LARGE{\textbf{\ExpTitle}}
\end{center}

% ===============================================================
% --------------------------- 实验目的 ---------------------------
% ===============================================================

\large{\textbf{【实验目的】}}%-----【实验目的】-----

\begin{enumerate}
    \item %TODO:实验目的
\end{enumerate}
    
% ===============================================================
% --------------------------- 仪器用具 ---------------------------
% ===============================================================

\large{\textbf{【仪器用具】}} %-----【仪器用具】-----

\begin{center}
    \begin{tabular}{|c|c|c|p{8cm}|}%{|c|c|c|c|}%{|p{2cm}|p{4cm}|p{4cm}|p{4cm}|}
        \hline
        编号 & 仪器用具名称 & 数量 & 主要参数 \\
        \hline
        1 &  & 1 &  \\ %TODO:仪器用具
        \hline
        2 &  & 2 &  \\
        \hline
        3  &   & 1 &  \\
        \hline
        4  &   & 1 &  \\
        \hline
    \end{tabular}
\end{center}

% ===============================================================
% ------------------------- 实验注意事项 -------------------------
% ===============================================================

\large{\textbf{【实验注意事项】}} %-----【实验注意事项】-----

\begin{enumerate}
    \item 
\end{enumerate}

% ===============================================================
% --------------------------- 原理概述 ---------------------------
% ===============================================================

% \large{\textbf{【原理概述】}} %-----【原理概述】-----
\section{原理概述}
djskaldjkla的算啦\par
但撒开奖
但撒开奖
但撒开奖
但撒开奖
但撒开奖
但撒开奖
但撒开奖
但撒开奖
但撒开奖
但撒开奖
但撒开奖
但撒开奖
但撒开奖
但撒开奖
但撒开奖
但撒开奖
但撒开奖
但撒开奖
但撒开奖
但撒开奖
但撒开奖
但撒开奖
但撒开奖
但撒开奖
但撒开奖
但撒开奖
但撒开奖
% 

% ===============================================================
% ------------------------- 实验前思考题 -------------------------
% ===============================================================
\section{实验前思考题}
\begin{enumerate}
    \item 思考题1
    \begin{ans}
        答案一
    \end{ans}
    \item 思考题2
    \item 思考题3
\end{enumerate}


% !!!!!============================!!!!!!!============================!!!!!
% !!!!!~~~~~~~~~~~~~~~~~~~~~~~~~~~~~~~~~~~~~~~~~~~~~~~~~~~~~~~~~~~~~~~!!!!!
% !!!!!--------------------------- 实验记录 ---------------------------!!!!!
% !!!!!~~~~~~~~~~~~~~~~~~~~~~~~~~~~~~~~~~~~~~~~~~~~~~~~~~~~~~~~~~~~~~~!!!!!
% !!!!!============================!!!!!!!============================!!!!!
\newpage

\begin{center}
    \begin{tabular}{|p{2cm}|p{4cm}|p{4cm}|p{4cm}|}
        \hline
        专业: & 物理学 & 年级: & 17级 \\
        \hline
        姓名: & \stuName & 学号: & \stuID \\
        \hline
        日期: &  & & \\
        \hline
        评分:  &   & 教师签名:& \\
        \hline
    \end{tabular}
\end{center}

\begin{center}
    \LARGE{\textbf{\ExpTitle}}
\end{center}

% ===============================================================
% --------------------- 实验内容、步骤、结果 ----------------------
% ===============================================================

% \large{\textbf{【实验内容、步骤、结果】}} %-----【实验内容、步骤、结果】-----
\section{实验内容、步骤、结果}


% ===============================================================
% --------------------- 实验过程遇到问题记录 ----------------------
% ===============================================================

\large{\textbf{【实验过程遇到问题记录】}} %-----【实验过程遇到问题记录】-----

% !!!!!============================!!!!!!!============================!!!!!
% !!!!!~~~~~~~~~~~~~~~~~~~~~~~~~~~~~~~~~~~~~~~~~~~~~~~~~~~~~~~~~~~~~~~!!!!!
% !!!!!--------------------------- 分析讨论 ---------------------------!!!!!
% !!!!!~~~~~~~~~~~~~~~~~~~~~~~~~~~~~~~~~~~~~~~~~~~~~~~~~~~~~~~~~~~~~~~!!!!!
% !!!!!============================!!!!!!!============================!!!!!
\newpage

\begin{center}
    \begin{tabular}{|p{2cm}|p{4cm}|p{4cm}|p{4cm}|}
        \hline
        专业: & 物理学 & 年级: & 17级 \\
        \hline
        姓名: & \stuName & 学号: & \stuID \\
        \hline
        日期: &  & & \\
        \hline
        评分:  &   & 教师签名:& \\
        \hline
    \end{tabular}
\end{center}


\begin{center}
    \LARGE{\textbf{\ExpTitle}}
\end{center}

% \setcounter{section}{0}

% ===============================================================
% ------------------------- 分析与讨论 ---------------------------
% ===============================================================

% \large{\textbf{【分析与讨论】}} %-----【分析与讨论】-----
\section{分析与讨论}


% ===============================================================
% ------------------------ 实验后思考题 --------------------------
% ===============================================================

% \large{\textbf{【实验后思考题】}} %-----【实验后思考题】-----
\section{实验后思考题}

% \newpage

\begin{appendices}
    \section{Appendix A}
\end{appendices}


\end{document}