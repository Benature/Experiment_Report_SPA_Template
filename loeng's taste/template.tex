\documentclass[no-math,zihao = -4]{ctexart} %由于前置宏包的需求,强烈建议使用ctexart且加上"no-math,zihao = -4"的设置。

%====其他有用的宏包====%
\usepackage{tabularx} %画均分表格
\usepackage{amsmath} %数学
\usepackage{amssymb}
\usepackage{bm} %便捷使用数学模式中的字母
\usepackage{mathrsfs}%使用花体字 命令\mathscr
\usepackage{subfigure}
\usepackage{booktabs}

%====使用字体====%
\setCJKmainfont[BoldFont = {Songti SC Black}]{Songti SC} %中文字体使用宋体
\setmainfont{Times New Roman} %英文字体设置为Times New Roman
\usepackage{euler} %使用euler字体

%====ctex文档设置====%
\ctexset{
    punct = banjiao, %使用半角标点符号
}

%====geometry版面设置====%
\usepackage{geometry}
\geometry{
    scale = {0.8,0.8}, %layout区域上下边界与左右边界占纸张的0.8
    centering %layout居中
}

%====页眉/脚设置====%
\usepackage{lastpage}
\usepackage{fancyhdr}
\pagestyle{fancy}

\fancyhf{} %清空默认的页眉/脚

\fancyhead[L]{\heiti \footnotesize \exptitle} %左页眉使用小五黑体显示实验标题。
\fancyhead[R]{\heiti \footnotesize \leftmark} %右页眉显示section
\fancyfoot[C]{\heiti \footnotesize 第\textbf{\thepage}页\\共 \pageref*{LastPage} 页} %中页脚显示页数


%====浮动体标题====%
\usepackage{floatrow} %全局设置图标题在下、表标题在上。
\floatsetup[table]{capposition=top}
\floatsetup[figure]{capposition=bottom}

\usepackage{caption} %对于标题属性如:字体、前缀、分隔的设置
\captionsetup{
    labelformat = parens,
    labelsep = colon,
    tableposition = bottom,
    figureposition = bottom,
    font = small
}


%====交叉引用格式====%
\usepackage{hyperref} %一个强大且多元的宏包,交叉引用前的前缀名,由于某些原因,只能想到蠢方法去定义。
\renewcommand{\figureautorefname}{\figurename}
\renewcommand{\tableautorefname}{\tablename}
\renewcommand{\partautorefname}{\partname}
\renewcommand{\appendixautorefname}{\appendixname}
\renewcommand{\equationautorefname}{\equationname}
\renewcommand{\itemautorefname}{\itemname}
\renewcommand{\chapterautorefname}{\chaptername}
\renewcommand{\sectionautorefname}{\sectionname}
\renewcommand{\subsectionautorefname}{\subsectionname}
\renewcommand{\subsubsectionautorefname}{\subsubsectionname}
\renewcommand{\paragraphautorefname}{\paragraphname}
\renewcommand{\Hfootnoteautorefname}{\Hfootnotename}
\renewcommand{\AMSautorefname}{\AMSname}
\renewcommand{\theoremautorefname}{\theoremname}
\renewcommand{\pageautorefname}{\pagename}

\hypersetup{
    hidelinks, %隐藏超链接的炫酷色彩
    CJKbookmarks = true, %启用书签
    bookmarksopen = true, %字面上是书签展开,但真实功能未知
    bookmarksnumbered = true %书签标注编号(按计数器标注)
}

%====公式====%
\numberwithin{equation}{section}% 公式按章节分


%====特制环境====%
\usepackage[dvipsnames]{xcolor}
\usepackage[many]{tcolorbox}

\newtheorem{ans}{回答}[enumi] %answer环境
\tcolorboxenvironment{ans}{
  boxrule=0pt,
  boxsep=2pt,
  colback={White!90!Cerulean},
  enhanced jigsaw, 
  borderline west={2pt}{0pt}{Cerulean},
  sharp corners,
  before skip=10pt,
  after skip=10pt,
  breakable,
}

\newtheorem{conc}{结论}[section] %conclusion
\tcolorboxenvironment{conc}{
  boxrule=0pt,
  boxsep=2pt,
  colback={White!90!Dandelion},
  enhanced jigsaw, 
  borderline west={2pt}{0pt}{Dandelion},
  sharp corners,
  before skip=10pt,
  after skip=10pt,
  breakable,
}


%====预设命令====%
\newcommand{\at}[1]{\renewcommand{\arraystretch}{#1}} %做表格时调节行距便捷一些
\newcommand{\ct}{\centering} %居中更迅速

%====待填信息,这部分内容请使用时据实际修改第二个大括号的内容====%
\newcommand{\exptitle}{CC1+A 热辐射的测量设计性实验} %建议输入完整的实验名称
\newcommand{\stid}{1XXXXXXX} %输入自己的学号
\newcommand{\major}{XX学} %填入专业信息:如"物理学"
\newcommand{\grade}{20XX级} %年级
\newcommand{\name}{张三} %你的名字


%====正文处模版(非固定样式)====%
\begin{document}

\begin{center}
    \at{1.8}
    \begin{tabularx}{0.9 \textwidth}{|*{8}{>{\centering \arraybackslash}X|}}
        \specialrule{0em}{0.3cm}{0cm}
        \hline
        \multicolumn{2}{|c|}{\Large 预习实验} & \multicolumn{2}{c|}{\Large 实验记录}& \multicolumn{2}{c|}{\Large 分析讨论} & \multicolumn{2}{c|}{\Large 总成绩} \\
        \hline
        &&&&&&& \\
        \hline
        \specialrule{0em}{0.3cm}{0cm}
    \end{tabularx}

    \at{1.3}
    \begin{tabularx}{0.9 \textwidth}{|p{2cm}<{\ct}|p{4.5cm}<{\ct}|p{3cm}<{\ct}|X<{\ct}|}
        \hline
        {专业}: & {\major} & {年级:}   & {\grade} \\
        \hline
        {姓名:} & {\name}  & {学号:}   & {\stid} \\
        \hline
        {日期:} & {\today} & {教师签名:}& \\
        \hline
        \specialrule{0em}{0.6cm}{0cm}
    \end{tabularx}
\end{center}

\begin{center}
    \Large \exptitle
\end{center}


%前置预设表格结束

\section{背景介绍}
    按个人习惯这里会放一些如保密信息发展史之类、半导体对科技发展对贡献之类的话语。小小篇幅不成敬意。
    
    \subsection{实验目的}
        \begin{enumerate}
            \item 第一个目的
            \item 第二个阴谋
        \end{enumerate}

    \subsection{实验仪器}
        \begin{center}
            \begin{tabular}[c]{cccc}
                \toprule
                    编号   &仪器用具名称 &  数量   &主要参数(型号,规格等)\\ 
                \midrule
                    1    & 准直激光器   & 2 & 波长:404nm,最大功率:150mW  \\ 
                    2    & 偏振分光棱镜  & 2 & 波长:404nm,消光比>500 \\ 
                \bottomrule
            \end{tabular}
            \captionof{table}{量子叠加与经典混合差异的用具}
            \label{tab:实验用具}
        \end{center}

    \subsection{实验注意事项}
        \begin{enumerate}
            \item 注意人生安全
            \item 记得吃饭
            \item 记得睡觉
            \item 记得喝水
            \item 记得呼吸
        \end{enumerate}
\section{原理概述}
\section{实验步骤}






\end{document}